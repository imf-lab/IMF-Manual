\documentclass[../main.tex]{subfiles}
\graphicspath{{\subfix{../}}}
\begin{document}

\chapter{Preface}
The development of the Information Modelling Framework (IMF) was progressed through the
READI\footnote{\url{https://readi-jip.org/}} joint industry project, resulting in an IMF concept document issued in March
2021. Following this, the development continued as part of the Equinor's Krafla and Wisting projects, and then was
extended to include the partners of the NOAKA digital cooperation, Equinor and Aker BP. The work has from the beginning been conducted in collaboration with the SIRIUS Centre at the
University of Oslo. 

In parallel with the continued development of the IMF, the
implementation of IMF was pioneered by the Krafla project together with Aibel, being the engineering contractor.
The learnings from this piloting work has contributed significantly to enhancing IMF and documenting IMF in the form of a
reference manual that comprises documentation and specifications that together answer what a company must do to
implement and use IMF.

It is intended that the results shall form the necessary basis to create a DNV recommended practice
    on how to implement IMF in the industry. Responding to an increasing interest and involvement from industry, a long term objective is make the results a basis for international standardisation.

\chapter{Introduction}
The Information Modelling Framework (IMF) establishes a new way of working using information models through the capture and logical flow of information during an industrial design and construct project through to operations, maintenance and disposal.

As the current way of working, 
actors in the industry use different methods, tools, and work processes to develop a facility asset. This misalignment is seen in the Capital Value Process (CVP), along the supply chain, and between disciplines and systems within the same facility asset. The consequences are loss of information, risk of safety and quality breaches, manual mappings and duplication of work and data, reduced possibilities for re-use of concepts and design, lock-in to a portfolio of applications, and a lot of company-specific requirements that are tuned to fulfil needs for a certain portfolio of applications.

The goal of IMF is to enable more reliable and efficient communication between organisations, people, and IT systems.

IMF is a method, a framework, and a language that allows creating an engineering friendly formal description of a facility asset, using graphical figures and common industry reference data libraries containing definitions of elements that are frequently re-used. The resulting information model of the facility asset contains information in a format which is readable to humans as well as to computers.

The overall requirements to functionality of the facility asset are captured in an initial information model. As the work progresses into the design phase of the facility asset, several more detailed information models are created by discipline expertise to mature the design. Due to the inherent features of the (machine-readable) format proposed in IMF, the information models can be continuously checked for design flaws by an automated work process. Furthermore, the fragments of information, now represented through several information models that are aligned through the use of IMF, can be integrated along the way to ensure a consistent design and a valid description of the facility asset on a holistic level.

When handover to operator takes place the information model, now a model-of-models, contains records of the historical context and design decisions. This holistic description and the contents of the facility asset is available at any level of detail, as required to operate, control, maintain, and later do modifications on the facility asset. Furthermore, access to information is not restricted by documents and formatting, but can be navigated freely and maintained efficiently.


\end{document}
