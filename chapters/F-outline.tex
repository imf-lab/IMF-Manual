\documentclass[../main.tex]{subfiles}
\graphicspath{{\subfix{../}}}
\begin{document}

\chapter{Outline and Readers Guide}

%This chapter introduces the vision of a new way of working with information models. \autoref{sec:userinformationmodel} gives a brief introduction to the IMF concept and new ways of working. \autoref{sec:newmethodsroles} discusses new methods, roles, and competencies that IMF introduces.

This document  introduces IMF and explains how IMF can be used to enable new ways of working for
engineering of facility assets, from the design begins and until the facility asset is decommissioned. 


\autoref{ch:Chapter 1} states the scope of this document.

\autoref{ch:Chapter 3} contains a brief description of the problem that IMF is developed to solve and a vision for how IMF can be implemented, including an overview of the
foreseen ecosystem of applications.
This chapter situates IMF in the context of its intended use. 

%An introduction to the IMF concept and how this solves the problem statement in \autoref{ch:Section 1.1} is presented in 


\autoref{ch:Chapter 2} introduces fundamental concepts underlying IMF. This chapter is necessary for understanding IMF in depth, but it is not required for a first reading. 

\autoref{ch:IMF Language: Overview} gives a complete overview of the IMF language. 

\autoref{ch:How to Create an IMF Model} is a guide for how to use IMF to
model facility assets.

\autoref{ch:How to Specify IMF Types} is a guide for how to create the building blocks for IMF models. 
%

%\autoref{ch:The IMF Language Formalized} formalises the IMF language and
% gives an overview of the different elements, relations, and rules in IMF. This 
% chapter is written for readers with prior knowledge of semantic technology. 
%

%is given in \autoref{ch:Chapter 7}.  At the end of the document a number of appendices covering topics of IMF that might be of interest for certain readers are provided.

For intended users of IMF the entire document should be read in order to get the necessary understanding of IMF and recommendation to how to start
modelling facility assets.

For readers wanting an overall understanding of the concept of IMF, reading \autoref{ch:Chapter 1}, \autoref{ch:Chapter 3} and \autoref{ch:Chapter 2}
should be sufficient. 

% Optionally, \autoref{ch:Chapter 7} can be read to get an understanding of how an implementation of an IMF eco-system is intended to work.

\end{document}
