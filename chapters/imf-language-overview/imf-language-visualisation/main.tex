\documentclass[11pt,fleqn]{article}

% from commons
\usepackage{commons/imf-spec}
\usepackage{commons/imf-spec-draft}

\usepackage{imf-style-tikz}
\usepackage{style}

%% avoid compiling heavy things
%\usepackage{environ}
%\RenewEnviron{figure}{}
%\RenewEnviron{table}{}
%\RenewEnviron{tblr}{[draft-mode]}

\newcommand{\thetitle}{IMF Language: Guidelines for Visualizing IMF Models}

\hypersetup{
 pdftitle={\thetitle},
 pdfkeywords={},
 pdfsubject={},
 pdflang={English}}

\begin{document}

%% Document metadata

\thispagestyle{empty}
{\bf\LARGE
  \thetitle
}
\vspace{3em}

\begin{description}
  
\item[{Version}] 0.2.2-alpha
  
\item[Date] \today
  
\item[Editors]
  Martin G. Skjæveland (\texttt{martige@uio.no})
  
\item[{Repository}] \url{https://gitlab.com/imf-lab/spec/imf-language-visualisation}

\item[Revision] \gitrevision
\end{description}

\section*{Abstract}

\input{README.md}

\pagebreak

\setcounter{tocdepth}{2}
\tableofcontents


\section{Introduction}

This specification gives guidelines for how the Information Modelling
Framework (IMF) language may be visualized.

\input{README.md}

The graphical notation is designed to be supported by common graph
model based languages and tools such as
RDF,\footnote{\url{https://www.w3.org/RDF/}}
GraphML,\footnote{\url{https://graphml.graphdrawing.org/}} and
Graphviz,\footnote{\url{https://graphviz.org/}} but should
be applicable to a wide range of modelling and visualization tools.

% \paragraph{Other specifications}

\RFCkwnote{}

\paragraph{Tools and implementations}

Information on tools and implementations of this specification can be
found in the repository of the specification.

\paragraph{Notation}

Terms defined in the formal IMF language are in this specification
written using the following font face: \ow{Element}, \ow{connectedTo}.

\section{Overview}

IMF models are represented visually using a graph model. \ow{Element}s
are represented as nodes, where the different types of \ow{Element}s,
e.g., \ow{Block}, \ow{Terminal} and \ow{Connector}, are represented
using different node shapes and sizes. Relationships between \ow{Element}s
are represented as edges between nodes, where different relations are
visualized using different arrowheads and line styles. \ow{Aspect}s
are represented using different fill colours.
\ow{Element}s can be grouped and the visualization of the group
can use visual effects to describe its \ow{Element}s.

\begin{figure}
  \centering
  \begin{tblr}{
      colspec = {Q[c,h]Q[c,h]Q[c,h]},
      stretch = 0,
      rowsep = 3pt,
    }
    Shapes & Relations & Fill colours \\ \hline
    \begin{tblr}{
        colspec = {Q[l,m]Q[c,m]},
        rowsep = 3pt,
      }
      \ow{Block}           & \tikz\node[block,scale=0.5]{};\\
      \ow{Connector}       & \tikz\node[connector]{};\\
      \ow{Terminal}        & \tikz\node[terminal]{};\\
      \ow{InputTerminal}   & \tikz\node[inputterminal]{};\\
      \ow{OutputTerminal}  & \tikz\node[outputterminal]{};\\
      \ow{BiTerminal}      & \tikz\node[biterminal]{};
    \end{tblr}
    &
    \begin{tblr}{
        colspec = {Q[l,m]Q[c,m]},
        rowsep = 3pt,
      }
      Topology        & \tikz{\draw (0,0) edge[connectedTo] (1,0);}\\
      Media Transfer  & \tikz{\draw (0,0) edge[transfersTo] (1,0);}\\
      Partonomy       & \tikz{\draw (0,0) edge[containedIn] (1,0);}\\
      Specialization  & \tikz{\draw (0,0) edge[specializationOf] (1,0);}\\
      Fulfills        & \tikz{\draw (0,0) edge[fulfills] (1,0);}\\
      Proxy           & \tikz{\draw (0,0) edge[proxy] (1,0);}\\
      Projection      & \tikz{\draw (0,0) edge[projection] (1,0);}\\
      Equality        & \tikz{\draw (0,0) edge[equals] (1,0);}
    \end{tblr}
    &
    \begin{tblr}{
        colspec = {Q[l,m]Q[c,m]},
        rowsep = 3pt,
      }      
      \ow{Activity}                 & \mybox{ffff00}\\
      \ow{Space}                    & \mybox{ff00ff}\\
      \ow{Implementation}           & \mybox{00ffff}\\
      No \ow{Aspect}                & \mybox{ffffff}\\
      Unspecified \ow{Aspect}       & \mybox{cccccc}
    \end{tblr}
  \end{tblr}
  \caption{\label{fig:graphics-all}All graphical elements.}
\end{figure}

\section{\ow{Element}s}
 
\ow{Element}s are visualized using nodes of different shapes; all node
shapes are displayed in \autoref{fig:graphics-node-shapes}. Nodes may have
different shapes and sizes. They may also be coloured; this is
described in \autoref{sec:aspect-elements}.
%
An \ow{Element} \kw{should} be labelled with an identifier or name when this
is necessary to distinguish it from other \ow{Element}s.

\begin{figure}
  \begin{center}
    \begin{tblr}{
        colspec = {Q[c]Q[c]Q[c]Q[c]Q[c]Q[c]},
        rowsep = 3pt
      }      
      \tikz\node[block]{}; &
      \tikz\node[connector]{}; &
      \tikz\node[terminal]{}; &
      \tikz\node[inputterminal]{}; &
      \tikz\node[outputterminal]{}; &
      \tikz\node[biterminal]{};
      \\
      \ow{Block} &
      \ow{Connector} &
      \ow{Terminal} &
      \ow{InputTerminal} &
      \ow{OutputTerminal} &
      \ow{BiTerminal}
      \\
    \end{tblr}
  \end{center}
\caption{\label{fig:graphics-node-shapes}All node shapes.}
\end{figure}


\subsection{\ow{Block}}

A \ow{Block} \kw{must} be visualized as a rectangle, and \kw{should} have square
corners. The rectangle \kw{should} be oriented so that the horizontal lines
are about twice the length of its vertical lines.

\subsection{\ow{Connector}}

A \ow{Connector} \kw{must} be visualized by a circle. The diameter of the circle
\kw{should} be about half the size of the vertical lines of the \ow{Block}s it
is topologically adjacent to.

\subsection{\ow{Terminal}}

A \ow{Terminal} \kw{must} be visualized by a square, and \kw{should} have rounded
corners.  

An \ow{InputTerminal} \kw{must} be labelled with a minus character.
An \ow{OutputTerminal} \kw{must} be labelled with a plus character.
A \ow{BiTerminal} \kw{may} be labelled with a minus and a plus character.
A \ow{Terminal} where the flow direction is unknown is unlabelled.

The \ow{Terminal} node shape \kw{should} be slightly smaller than the
\ow{Connector} it is topologically adjacent to.

A \ow{Terminal} \kw{should} be placed close to, or may even overlap with, its
\ow{Block} and \kw{should} always be placed closer to its \ow{Block} than
to its topologically adjacent \ow{Connector}.

An \ow{InputTerminal} \kw{should} be placed left of its \ow{Block}, and an
\ow{OutputTerminal} \kw{should} be placed right of its \ow{Block}.  A
\ow{BiTerminal} \kw{should} be placed at the side which reflects its most
prominent direction, or, if it has no prominent direction, below or above its \ow{Block}.

\section{Relations}

Relationships are visualized by edges (or lines or arrows) between
two nodes. An edge may have an arrowhead to indicate the
direction---no arrowhead if there is no natural direction, and may have different line style, e.g.,
solid, dashed or dotted. Edges may use different colours for
grouping, highlighting or hiding information, but colours on edges have
no formal meaning. All edge styles are displayed in \autoref{fig:graphics-all-edges}.

Note that we often do not distinguish visually between a relation and
its subrelations. This is to reduce the number of different visual
features. Edges \kw{may} be labelled with the name of their
relation.

\begin{figure}
  \centering
\begin{tblr}{
  colspec = {QQ[c]QQ[c]},
  stretch = 0,
  rowsep = 10pt,
  column{2} = {leftsep=1pt,rightsep=2cm},
  %hlines = {red5, 1pt},
  %vlines = {red5, 1pt},
}
$\ow{connectedTo}(a,b)$  &
\begin{tikzpicture}[baseline=(current bounding box.center)]
  \node (a) at (0,0)  {$a$};
  \node (b) at (2,0)  {$b$};
  \draw (a) edge[connectedTo] (b);
\end{tikzpicture}
&
$\ow{transfersTo}(a,b)$  &
\begin{tikzpicture}[baseline=(current bounding box.center)]
  \node (a) at (0,0)  {$a$};
  \node (b) at (2,0)  {$b$};
  \draw (a) edge[transfersTo] (b);
\end{tikzpicture}
\\
$\ow{containedIn}(a,b)$  &
\begin{tikzpicture}[baseline=(current bounding box.center)]
  \node (a) at (0,0)  {$a$};
  \node (b) at (0,2)  {$b$};
  \draw (a) edge[containedIn] (b);
\end{tikzpicture}
&&
\\
$\ow{specializationOf}(a,b)$  &
\begin{tikzpicture}[baseline=(current bounding box.center)]
  \node (a) at (0,0)  {$a$};
  \node (b) at (0,2)  {$b$};
  \draw (a) edge[specializationOf] (b);
\end{tikzpicture}
&
$\ow{fulfills}(a,b)$  &
\begin{tikzpicture}[baseline=(current bounding box.center)]
  \node (a) at (0,0)  {$a$};
  \node (b) at (0,2)  {$b$};
  \draw (a) edge[fulfills] (b);
\end{tikzpicture}\\
$\ow{proxy}(a,b)$  &
\begin{tikzpicture}[baseline=(current bounding box.center)]
  \node (a) at (0,0)  {$a$};
  \node (b) at (1.5,1.5)  {$b$};
  \draw (a) edge[proxy] (b);
\end{tikzpicture}
&
$\ow{projection}(a,b)$  &
\begin{tikzpicture}[baseline=(current bounding box.center)]
  \node (a) at (0,0)  {$a$};
  \node (b) at (1.5,-1.5)  {$b$};
  \draw (a) edge[projection] (b);
\end{tikzpicture}\\
$\ow{equals}(a,b)$  &
\begin{tikzpicture}[baseline=(current bounding box.center)]
  \node (a) at (0,0)  {$a$};
  \node (b) at (1.5,1.5)  {$b$};
  \draw (a) edge[equals] (b);
\end{tikzpicture}
&&
\end{tblr}
  \caption{All edge types, with their preferred orientation indicated.}
  \label{fig:graphics-all-edges}
\end{figure}

\subsection{Topology: \ow{connectedTo}}

Topology is represented by the relation
\ow{connectedTo} and its subrelations.

All topology relationships \kw{must} be visualized by a solid line with no
arrowhead, and \kw{should} be oriented horizontally.


\subsection{Media Transfer: \ow{transfersTo}}

Media transfer is represented with the relation \ow{transfersTo}.

These relationships \kw{must} be visualized by a solid line, in the
same style as topology relationships, but with a solid arrow head
pointing from the source and to the target of the media transfer, and
\kw{should} be oriented horizontally, placing sources left of targets.

\subsection{Partonomy: \ow{containedIn}}

Partonomy is represented with the relation \ow{containedIn}, its inverse relation and their subrelations.

These relationships \kw{must} be visualized by a solid line and a solid diamond
arrowhead pointing from the part and to the whole,
and \kw{should} be oriented vertically, placing parts below wholes.

\subsection{Proxy: \ow{proxy}}

Proxy relationships are represented by the relation \ow{proxy}.

These relationships \kw{must} be visualized by a dotted line without arrowhead, and
\kw{should} be oriented so that proxy elements appear as clusters, i.e., placed close to each other.

\subsection{Projection: \ow{projection}}

Projection relationships go between non-perspective elements and their proxies,
and are represented by the relation \ow{projection}.

These relationships \kw{must} be visualized by a dotted line, in the same
style as proxy relationships, but with a solid arrowhead, and
\kw{should} be oriented so that the source of the projection appears to be the central node, either by placing it in the
centre of or above the cluster or proxies.

\subsection{Specialization: \ow{specializationOf}}

Specialization relationships are represented with the relation
\ow{specializationOf} and its inverse.

These relationships \kw{must} be visualized by a solid line with an
open arrowhead pointing from the specific to the generic \ow{Element}, and
\kw{should} be oriented vertically, placing specific elements below generic.

\subsection{Requirement--Solution: \ow{fulfills}}

Requirement--solution relationships between \ow{Element}s express that
the target \ow{Element} is a solution to the requirement stated by
the source element.

Requirement--solution relationships are represented by the relation
\ow{fulfilledBy} and its inverse \ow{fulfills}.

These relationships \kw{must} be visualized by a dashed line
with an open arrowhead pointing from the solution element to the
requirement element, and
\kw{should} be oriented vertically, placing solution \ow{Element}s below requirement \ow{Element}s.

\ow{fulfilledBy} relationships are represented as the inverse of
\ow{fulfills}.


\subsection{Equality: \ow{sameAs}}

Sometimes it is necessary or convenient to represent one \ow{Element}
with multiple nodes. This \kw{should} be avoided due to the
added complexity this adds to understanding visualizations.
%
Equality relationships can be used for this purpose as they to express
that two \ow{Element}s are equal.

Equality relationships are represented by the relation \ow{sameAs}.

These relationships \kw{should} be visualized by an emphasized solid line,
e.g., the line can be thicker than other lines, or it can be drawn using
a double line. The line \kw{should} be labelled with an equals sign or with
the label ``\ow{sameAs}''.

\section{\ow{Aspect}s}
\label{sec:aspect-elements}

\ow{Aspect}s \kw{must} be visualized using a specified fill colour on the \ow{Element}s' shape.

Colours are given in \autoref{tbl:graphics-all-colours},
%
and \autoref{fig:graphics-ex-coloured-blocks} displays the colours demonstrated on \ow{Block}s.


\begin{figure}
\centering
\begin{tabular}{lll}
  \ow{Aspect}/\ow{Entity type} filter & Colour name & Colour Code\\
  \hline
  No \ow{Aspect} & White & \myboxt{ffffff}\\
  Unspecified \ow{Aspect} & Grey & \myboxt{cccccc}\\
  %
  \ow{Activity} & Yellow & \myboxt{ffff00}\\
  \ow{Space} & Magenta & \myboxt{ff00ff}\\
  \ow{Implementation} & Cyan & \myboxt{00ffff}\\
\end{tabular}
\caption{\label{tbl:graphics-all-colours} Predefined colours for \ow{Entity type} filters and \ow{Aspect}s.}
\end{figure}


\begin{figure}
  \begin{center}
    \begin{tblr}{
        colspec = {Q[c]Q[c]Q[c]Q[c]Q[c]Q[c]},
        rowsep = 3pt
      }
      \tikz\node[block,nonaspect]{}; &
      \tikz\node[block,unknown]{}; &
      \tikz\node[block,activity]{}; &
      \tikz\node[block,space]{}; &
      \tikz\node[block,implementation]{}; 
      %\\
      %\ow{ActivityBlock} &
      %\ow{ImplementationBlock} &
      %\ow{SpaceBlock} &
      %\ow{NonPerspectiveBlock} &
      %\ow{Block}
    \end{tblr}
  \end{center}
\caption{\label{fig:graphics-ex-coloured-blocks} Predefined colours demonstrated on \ow{Block}s.}
\end{figure}


The colour of additional \ow{Aspect}s \kw{should} be created by mixing the predefined colours so that it reflects the definition of the \ow{Aspect}.
For instance, an \ow{Aspect} whose primary focus is \ow{Activity} should be given a colour that is more yellow than magenta or cyan.
\autoref{tbl:graphics-colour-sugggestions} contains a list of suggested colours.

\begin{figure}
  \centering
  \begin{tabular}{ccc}
    \begin{tabular}{rrrlpp}
      Act & Spa & Imp & Colour code\\
      \hline
      1 &   &   & \myboxt{ffff00}\\
        & 1 &   & \myboxt{ff00ff}\\
        &   & 1 & \myboxt{00ffff}\\
      \hline
      1 & 1 &   & \myboxt{ff8080}\\
      1 &   & 1 & \myboxt{80ff80}\\
        & 1 & 1 & \myboxt{8080ff}\\
      \hline
      1 & 1 & 1 & \myboxt{aaaaaa}\\
      \hline
      2 & 1 &   & \myboxt{ffaa55}\\
      2 & 1 & 1 & \myboxt{bfbf80}\\
      2 &   & 1 & \myboxt{aaff55}\\
      1 & 2 &   & \myboxt{ff55aa}\\
      1 & 2 & 1 & \myboxt{bf80bf}\\
      1 & 1 & 2 & \myboxt{80bfbf}\\
      1 &   & 2 & \myboxt{55ffaa}\\
        & 2 & 1 & \myboxt{aa55ff}\\
        & 1 & 2 & \myboxt{55aaff}\\
    \end{tabular}
    &~~~~~~~&
    \begin{tikzpicture}[baseline=(current bounding box.center)]
      \begin{scope}[blend mode=multiply]
        \pgfsetfillopacity{0.4}
        \fill[cActivity]   ( 90:1.2) circle (1.5);
        \fill[cSpace] (210:1.2) circle (1.5);
        \fill[cImplementation]  (330:1.2) circle (1.5);
        \fill[cActivity]   ( 90:1.2) circle (2);
        \fill[cSpace] (210:1.2) circle (2);
        \fill[cImplementation]  (330:1.2) circle (2);
      \end{scope}
      \node at ( 70:2.5)    {\ow{Activity}};
      \node at ( 210:2.7)   {\ow{Space}};
      \node at ( 330:2.7)   {\ow{Implementation}};
    \end{tikzpicture}
  \end{tabular}
  \caption{\label{tbl:graphics-colour-sugggestions}Suggested colours for \ow{Aspect}s. The
    columns \textsf{Act}-ivity, \textsf{Spa}-ce and
    \textsf{Imp}-lementation indicate how many parts of this
    colour is added to create the suggested colour.}
\end{figure}


An \ow{AspectElement}s may be left unspecified in the case that their
\ow{Aspect} is clear from surrounding \ow{Element}s. This is in particular
the case for \ow{Terminal}s that share the \ow{Aspect} of the
\ow{Block} they belong to. \autoref{fig:unspecified-perspective} shows an example.

\begin{figure}
  \centering
  \begin{tikzpicture}
    \node[block,activity] (a) at (0,0)  {};
    \node[terminal,unknown] (b) at (2.5,0)  {};
    \draw (a) edge[connectedTo] (b);
  \end{tikzpicture}
  \caption{Example: A \ow{Terminal} with unspecified \ow{Aspect} will ``inherit'' the
    \ow{Aspect} of its \ow{Block}.}
  \label{fig:unspecified-perspective}
\end{figure}


\section{Groups}

Groups of \ow{Element}s are visualized by placing the \ow{Element}s in
an enclosing shape, typically a rectangle.
\autoref{fig:stages-group-ex} gives an example of grouping.

%%% 
% Groups are typically used for \ow{Element}s that share some
% characteristics can be used to avoid repeating visual representation
% of the shared values for each \ow{Element}, but rather apply the
% visual representation on the group visualization.

\begin{figure}
  \centering
  \begin{tikzpicture}[scale=.8]
  \node[block,unknown]     (a) at (0,0)   {$a$};
  \node[connector,activity] (c) at (4,0)   {$b$};
  \node[block,activity]     (e) at (8,0)   {$c$};
  \node[block,activity]     (f) at (8,-3)  {$d$};

  \node[fit=(a)(f), draw, inner sep=20pt] {};

  \draw (a) edge[connectedTo] (c);
  \draw (c) edge[connectedTo] (e);

  \draw (f) edge[containedIn] (e);

  \node[block,implementation]     (a2) at (0,-8)  {$m$};
  \node[terminal,unknown]        (b2) at (2,-8) {$n$};
  \node[connector,implementation] (c2) at (4,-8)  {$o$};
  \node[terminal,unknown]        (d2) at (6,-8)  {$p$};
  \node[block,implementation]     (e2) at (8,-8)  {$q$};

  \draw (a2) edge[connectedTo] (b2);
  \draw (b2) edge[connectedTo] (c2);
  \draw (c2) edge[connectedTo] (d2);
  \draw (d2) edge[connectedTo] (e2);

  \draw (b2) edge[transfersTo,bend left=30] (d2);
  
  \node[fit=(a2)(e2), draw, inner sep=20pt] {};

  \draw (a2) edge[fulfills] (a);
  \draw (e2) edge[fulfills] (f);
  
\end{tikzpicture}

\caption{\label{fig:stages-group-ex} Example of grouping. The
  visualization contains two groups: the \ow{Element}s labelled $a$,
  $b$, $c$ and $d$; and $m$, $n$, $o$, $p$ and $q$. 
  %, share the same
  %\ow{Stage} as indicated by the border colour of the box surrounding
  %them.
}
\end{figure}

\section{Other: Attributes, Classifiers and metadata}

The IMF language supports assigning more information to \ow{Element}s; this includes \ow{Attribute}s, \ow{classifier}s and metadata, such as textual definitions.

In order to keep visualizations of IMF models simple, these additional
information objects do not have a graphical representation. Rather, if
necessary, they can be represented as lists of attributes that
are associates with its \ow{Element} using appropriate features of the
used visualization tool or language.

\section{Types and Templates}

The IMF language supports creating IMF model statements by
instantiating types and templates. This version of the specification does not
describe how to represent the association between a type or template and its
instances.
 
\end{document}