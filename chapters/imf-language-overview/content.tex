

This specification gives a formal description of the Information
Modelling Framework (IMF) language.
%
Its purpose is to provide an abstract and implementation independent
specification of the IMF language, and with that give an introduction
to the language and provide a common basis for practical and
theoretical applications of IMF, such as specifying logical
formalizations, serialization formats, and software applications and
services that use IMF.

The specification is based on the developments of the following documents:
\begin{itemize}
\item \textbf{IMF Language: Logic Formalization}, which gives a
  first-order logic formalization of the IMF language.\\
  \url{https://gitlab.com/imf-lab/spec/imf-language-logic-formalisation}
\item \textbf{IMF Language: Guidelines for Visualizing IMF Models},
  which defines a graphical language for drawing IMF models with
  nodes, edges and colours.\\
  \url{https://gitlab.com/imf-lab/spec/imf-language-visualisation}
\item \textbf{IMF Language: Semantic Technologies Implementation},
  which specifies an implementation of the IMF language using the W3C
  semantic technology recommendations OWL, SHACL and RDF.\\
  \url{https://gitlab.com/imf-lab/spec/imf-ontology}
\end{itemize}

%\subsection{Scope}
%\label{sec:scope}


\subsection*{Notation}

This document uses the following notation.

\subsubsection*{Ontology}

The IMF vocabulary is written using the following typing:
\begin{itemize}
\item Classes are written using \ow{UpperCamelCase} and appear as
  unary predicates in examples.
\item Relations are written using \ow{lowerCamelCase} and appear as
  binary predicates in examples.
\item Individuals are written using \ow{lowerCamelCase} and appear as
  constants in examples.
\end{itemize}


\subsubsection*{Diagrams}

The diagrams that illustrate how classes, relations and
individuals in the IMF vocabulary are related are informative only
(and not normative). They indicate subclass relationships between
classes, membership relationships between individuals and classes, and
the domain and range of relations.

%\subsubsection{Visualisation}


\section{Overview}


The IMF language is split into the following main parts: ontology,
models, types and templates. Closely related to the IMF language are also
(object) identifiers. A high-level description of these parts
are described below and informally displayed in
\autoref{fig:imf-lang-parts}.

\begin{figure}
  \centering
  \tikzstyle{lab}=[fill=white]
  \tikzstyle{imf}=[fill=yellow!20]
  \tikzstyle{imf2}=[fill=yellow!5]
  \tikzstyle{oth}=[fill=gray!10]
  \begin{tikzpicture}[
    xscale=2.3,yscale=1.3,
    ]
    \sf

    \node[imf](o)  at (-3.5, 0) {Ontology};
    \node[imf](v)  at (-2,-1) {Vocabulary};
    \node[imf](a)  at (-2, 1) {\{Axiom\}};

    \node[draw,gray!15,fit=(o)(v)(a),inner sep=10pt] {};

    \node[oth](i)   at (0 ,-3.5) {Instance};

    \node[imf](te)   at (-2,-3.5) {Template};
    \node[imf](ty)   at (-1,-2.25) {Type};
    
    \node[imf2](id) at (1.5 ,-1.75) {Object Identifier};

    \node[oth](f)  at (0 , 0) {\{Fact\}};
    \node[imf](m)  at (2 , 0) {Model};

    \node[draw,gray!15,fit=(m)(f),inner sep=10pt] {};

    \draw[-latex] (o) -> node[lab]{consists of} (a);
    \draw[-latex] (o) -> node[lab]{consists of} (v);
    \draw[-latex,dotted] (a) -> node[lab]{defines} (v);

    \draw[-latex] (i) -> node[lab]{represents} (f);

    \draw[-latex,dashed] (te) -> node[lab]{ref.} (v);
    \draw[-latex,dashed] (ty) -> node[lab]{ref.} (v);
    \draw[-latex,dashed] (te) -> node[lab]{ref.} (ty);

    \draw[-latex,dotted] (i) -> node[lab]{instantiates} (te);
    \draw[-latex,dotted] (i) -> node[lab]{arguments} (id);

    \draw[-latex] (f) -> node[lab]{uses} (id);
    \draw[-latex] (f) -> node[lab]{ref.} (v);
    \draw[-latex] (f) -> node[lab]{ref.} (ty);

    \draw[-latex] (m) -> node[lab]{consists of} (f);

    \draw[-latex] (a) -> node[lab]{verifies} (f);

  \end{tikzpicture}
  \caption{The parts of the IMF language and how they are related.}
  \label{fig:imf-lang-parts}
\end{figure}

\subsection{IMF Ontology}

The ontology comprises 
\begin{itemize}
\item A vocabulary, i.e., a reserved set of terms, that is used to express
  facts about the target domain.
\item A set of rules over the vocabulary, called axioms, that
  describe the intended use of the vocabulary and hence how the
  vocabulary is used to express correct and meaningful facts.
\end{itemize}

The IMF ontology is described in \autoref{sec:ontology}.

\begin{example}\label{ex:imf-ontology}
  The IMF ontology contains vocabulary terms such as \ow{Block},
  \ow{Terminal}, \ow{partOf} and \ow{connectedTo}, and axioms such as
  ``A \ow{Terminal} can only be \ow{partOf} other \ow{Terminal}s'' and
  ``Blocks are not Terminals'', which can be formalized as follows
  with these logic formulas:
\begin{align*}
\forall x,y ( \ow{Terminal}(x) \land \ow{partOf}(x,y) &\rightarrow \ow{Terminal}(y))\\
\forall x ( \ow{Block}(x) &\rightarrow \neg \ow{Terminal}(x))
\end{align*}
\end{example}

\subsection{Identifiers}

In the context of the IMF language specification, an object identifier
is a name for an object or entity that is described using the IMF language.

A goal of IMF is to be able to clearly identify distinct descriptions
of assets, and to integrate data that is typically distributed across
multiple systems and diagrams using multiple identifiers.

%Identifiers are further described in \autoref{sec:identifiers}.

\begin{example}
  Examples of identifiers are tags, system identifiers and database keys.

  In our simple formal examples, we use $a$, $b$,
  $c$, $\ldots$ to represent identifiers.
\end{example}

\subsection{IMF Models}

An IMF model is a language construct for grouping and organizing facts
expressed using the IMF vocabulary, i.e., it is essentially just a set
of facts, a dataset. A fact is expressed using the vocabulary and
identifiers.

%IMF models are further described in \autoref{sec:model}.

\begin{example}
  The following is an IMF model:\\
  \begin{tabularx}{\textwidth}{@{}p{.6\textwidth}X}
  {\begin{align*}
    \{~&\ow{Block}(a), \ow{connectedTo}(a,e), \ow{Terminal}(e),\\
      &\ow{Block}(b), \ow{connectedTo}(b,f), \ow{Terminal}(f), \\
      &\ow{partOf}(b,a),\\
      &\ow{partOf}(f,e)~ \}
  \end{align*}}
  &
  \begin{tikzpicture}[baseline=1cm]
    \node[block]    (a) at (0,0)  {a};
    \node[terminal] (e) at (2,0) {e};
    \node[block]    (b) at (0,-2)  {b};
    \node[terminal] (f) at (2,-2)  {f};
    \draw[connectedTo] (a) -- (e);
    \draw[connectedTo] (b) -- (f);
    \draw[containedIn]      (b) -- (a);
    \draw[containedIn]      (f) -- (e);
  \end{tikzpicture}
  \end{tabularx}
  
The model states that $a$ and $b$ are \ow{Block}s, and $e$ and $f$ are
\ow{Terminal}s.
$a$ is \ow{connectedTo} $e$,
$b$ is \ow{connectedTo} $f$,
$b$ is \ow{partOf} $a$,
and
$f$ is \ow{partOf} $e$.
This model complies with the axioms in \autoref{ex:imf-ontology}.
The model is visualized using the graphics listed in \autoref{fig:graphics-all}.
\end{example}


\subsection{IMF Types and IMF Templates}

Informally, \emph{types} are used by SMEs to express typical configurations of model elements, in order to support standardization, reuse, and consistency and uniformity of information models. A type will typically represent a concept that is standard throughout an engineering discipline.

A \emph{template} is used to consistently create a specific configuration of model elements, in order to simplify model construction and maintenance. The configuration that a template represents need not---but can---be standard, and may be created to support modelling consistency and uniformity within only a specific engineering project.

Hence, types and templates are similar, but they differ in their scope; types are expected to serve a wider use than templates, while templates will on the other hand typically represent more detailed configurations, often orchestrating and configuring types for particular modelling use cases.
%
Instantiating types and templates from carefully curated libraries is a primary way of efficiently creating and maintaining information models of high quality.

This informal understanding of types and templates gives rise to the following more formal definition of IMF types and IMF templates.

An \emph{IMF type} is represented by an expression over the IMF
vocabulary that defines a class of all the objects that adhere to the
configuration specified by the IMF Type. These objects are called
instances of the type or type instances.
%
An \emph{IMF template} is a language construct for representing reusable
modelling patterns over the IMF vocabulary. A template instance is a
replica of the template's modelling pattern that follows the
conditions that may be specified by the template.  A template instance
represents a set of facts over the vocabulary. Hence, a template
instance represents a small IMF model.

There are hence clear similarities between an IMF type and an IMF
template.
The difference is that
an IMF type \emph{describes} a class of objects using the IMF vocabulary,
while
an IMF template is used to \emph{generate} a set of facts over the IMF vocabulary.
An IMF Template may refer to an IMF Type and hence
generate facts that are instances of IMF Types.

\begin{example}\label{ex:Block-with-Terminal-Type}
  A very simple (and artificial) IMF type is \ow{Block-with-Terminal-Type}:
  \begin{align*}
    \ow{Block-with-Terminal-Type}(x) \rightarrow  ( \ow{Block}(x) \land \exists y \ow{Terminal}(y) \land \ow{connectedTo}(x,y) )
  \end{align*}
  This IMF Type specifies the class of all \ow{Block}s that have a \ow{Terminal}.
\end{example}

\begin{example}
  A very simple IMF template is \ow{Block-with-Terminal-Template}:
  \begin{align*}
    \ow{Block-with-Terminal-Template}(x,y) \mapsto \{ \ow{Block}(x), \ow{Terminal}(y), \ow{connectedTo}(x,y) \}
  \end{align*}
  The $x$ and $y$ are the parameters of the template, and it
  represents the pattern of a \ow{Block} with a \ow{Terminal} connected
  to it.

  An instance of the template,
  $\ow{Block-with-Terminal-Template}(a,b)$, expands to the model:
  \begin{align*}
    %\ow{Block-with-Terminal}(a,b) \mapsto
    \{ \ow{Block}(a), \ow{Terminal}(b), \ow{connectedTo}(a,b) \}
  \end{align*}
  Note that the object $a$ is an instance of the IMF type \ow{Block-with-Terminal-Type} from \autoref{ex:Block-with-Terminal-Type}.
\end{example}



% \subsection{Example}

% @@TODO:
%  %
% Present a complete example that can act as a running example for the
% document?

% \section{Principles}

% @TODO

% The IMF language is designed to be used as a [take text from
% principles doc]

% The idea is that one system area domain should not result
% in multiple different IMF models.

% The IMF language designed according to the following principles:
% \begin{itemize}
% \item simple to use
% \item \ldots{}
% \end{itemize}

% As a consequence, the IMF language is kept deliberately small in size
% and contains a limited number of modelling constructs that (are ment
% to) describe systems along orthogonal dimensions. 


\section{Ontology}
\label{sec:ontology}

\subsection{Overview}

A vocabulary is a reserved set of terms which is split into
\emph{classes}, \emph{relations} and \emph{individuals}.

\begin{figure}
  \centering
  \begin{tblr}{
      colspec = {Q[c,h]Q[c,h]Q[c,h]},
      stretch = 0,
      rowsep = 3pt,
    }
    Shapes & Relations & Fill colours \\ \hline
    \begin{tblr}{
        colspec = {Q[l,m]Q[c,m]},
        rowsep = 3pt,
      }
      \ow{Block}           & \tikz\node[block,scale=0.5]{};\\
      \ow{Connector}       & \tikz\node[connector]{};\\
      \ow{Terminal}        & \tikz\node[terminal]{};\\
      \ow{InputTerminal}   & \tikz\node[inputterminal]{};\\
      \ow{OutputTerminal}  & \tikz\node[outputterminal]{};\\
      \ow{BiTerminal}      & \tikz\node[biterminal]{};
    \end{tblr}
    &
    \begin{tblr}{
        colspec = {Q[l,m]Q[c,m]},
        rowsep = 3pt,
      }
      Topology        & \tikz{\draw (0,0) edge[connectedTo] (1,0);}\\
      Media Transfer  & \tikz{\draw (0,0) edge[transfersTo] (1,0);}\\
      Partonomy       & \tikz{\draw (0,0) edge[containedIn] (1,0);}\\
      Specialization  & \tikz{\draw (0,0) edge[specializationOf] (1,0);}\\
      Fulfills        & \tikz{\draw (0,0) edge[fulfills] (1,0);}\\
      Proxy           & \tikz{\draw (0,0) edge[proxy] (1,0);}\\
      Projection      & \tikz{\draw (0,0) edge[projection] (1,0);}\\
      Equality        & \tikz{\draw (0,0) edge[equals] (1,0);}
    \end{tblr}
    &
    \begin{tblr}{
        colspec = {Q[l,m]Q[c,m]},
        rowsep = 3pt,
      }      
      \ow{Activity}                 & \mybox{ffff00}\\
      \ow{Space}                    & \mybox{ff00ff}\\
      \ow{Implementation}           & \mybox{00ffff}\\
      No \ow{Aspect}                & \mybox{ffffff}\\
      Unspecified \ow{Aspect}       & \mybox{cccccc}
    \end{tblr}
  \end{tblr}
  \caption{\label{fig:graphics-all}All graphical elements.}
\end{figure}

\subsubsection{Classes}

The single top-most class in the IMF vocabulary's class hierarchy is \ow{Entity};
everything is a \ow{Entity}.
\ow{Entity} is a very generic concept
%, and should not be directly instantiated,
%rather should instantiate one of its subclasses.
%\ow{Entity}
that is used by the ontology to state generic axioms that hold
for all objects described by the ontology.
\ow{Entity} should not be directly instantiated, rather instantiate an appropriate subclass of \ow{Entity}.



\subsubsection{Individuals}
\label{sec:individuals}

The ontology contains a limited number of individuals that represent
reserved words in the vocabulary, such as names for \ow{Aspect}s.
In this specification these individuals are presented as part of the presentation of the main class they are member of.


\subsubsection{Relations}

All relations are binary, i.e., all relationships relate two objects. As a
high-level and informal categorization of relations, we distinguish
between:

\begin{itemize}
\item \emph{Object} and \emph{datatype} relations. Object relations
  relate two objects, while datatype relations associate a
  data value, such as a string, a number or a date, to an object.

  % . This distinction is the same as for OWL's ObjectProperty and
  % DatatypeProperty [TODO cite]
\end{itemize}

For object relations the categorization also includes:

\begin{itemize}
\item \emph{Hierarchical} vs. \emph{associative} relations. Hierarchical relations
specify an ordering or ranking, while associative relations (or
non-hierarchical relations) do not specify a clear ordering and
often relate elements of similar rank (as specified by a
hierarchical relation).
\item \emph{Internal} vs. \emph{external} relations. Internal relations are defined
with a domain and range that is specified by IMF classes, while
\emph{external relations} do not and will usually relate to objects from
external reference data libraries.
\item \emph{Logical} vs. \emph{annotation} relations. Logical relations have formal
logical meaning which must be considered by logical reasoners, while
annotations are typically used for metadata such as creation date
and may be ignored by logical reasoners.
% This distinction is the same as for OWL's ObjectProperty and
% DatatypeProperty on the one hand, and AnnotationProperty on the
% other hand [cite].
\end{itemize}

%@TODO "This is not covered in this specification" \textbf{Metadata} are used for
%annotations, e.g., names, textual definitions, versioning and
%provenance data.

Unless otherwise noted, all relations are object type, associative,
internal and logical.


\subsection{Generic relations}

Membership and equality relationships are generic
relations that can be used to describe and reuse abstractions and
categorizations. They are applicable to all \ow{Entity}-s.


\subsubsection{Equality}

\ow{sameAs} is a relation between
\ow{Entity}-s, such that $\ow{sameAs}(a,b)$ expresses that
the \ow{Entity}-s $a$ and $b$ are equal,
i.e., $a$ and $b$ are different names/identifiers for the same $\ow{Entity}$.

\begin{axioms}~
  \begin{enumerate}[resume=axioms,{label=(\arabic*)}]
  \item \ow{sameAs} is a relation between \ow{Entity}-s.
  \item \ow{sameAs} is reflexive, symmetric, and transitive.
  \end{enumerate}
\end{axioms}

\subsubsection{Membership}
\label{sec:membership}

\ow{memberOf} is a relation between \ow{Entity}-s,
such that $\ow{memberOf}(a,b)$ expresses that the \ow{Entity} $a$ is a member of the \ow{Entity} $b$,
and hence that $b$ is a collection or group for $a$.
The inverse relation of \ow{memberOf} is \ow{hasMember}.

\begin{axioms}~
  \begin{enumerate}[resume=axioms,{label=(\arabic*)}]
  \item \ow{memberOf} is a relation between \ow{Entity}-s.
  \item \ow{hasMember} is the inverse of \ow{memberOf}.
  \end{enumerate}
\end{axioms}

\subsection{\ow{InformationArtefact}s}

\ow{Entity} has a single subclass \ow{InformationArtefact} that is the common class for the core categories of the ontology:
\ow{Element}, \ow{Descriptor} and \ow{InformationModel} (\autoref{uml-toplevel}).
These subclasses are pairwise disjoint.

\ow{Element} (\autoref{sec:Elements}) is the core class of the ontology. All classes and
relations in the ontology exist to describe \ow{Element}s---either directly or indirectly.

A \ow{Descriptor} is an \ow{InformationArtefact} that serves to describe or identify a feature of an \ow{Entity}.

A \ow{reference} is a generic relation that relates a resource to a \ow{Descriptor}.

\ow{Descriptor}s are associated with other \ow{Descriptor}s with the reference relation \ow{hasCharacteristic}.

\begin{figure}
  \centering
  \begin{tikzpicture}
  \tikzElement

  \node[oClass, right = 2 of e] (d)    {Descriptor};
  \node[oClass, right = 2 of d] (i)    {InformationModel};

  \node[oClass, above = of d]   (ia)   {InformationArtefact};
  \node[oClass, above = of ia]  (ent)  {Entity};

  \draw[oSpec] (e) -- (ia);
  \draw[oSpec] (d) -- (ia);
  \draw[oSpec] (i) -- (ia);
  \draw[oSpec] (ia) -- (ent);



  \draw[oRelDir,emph] (d) to [out=60 ,in=10 ,looseness=4] node[oRlab] {hasCharacteristic}   (d);

  \draw[oRelDir,emph] (i) to [bend left] node[oRlab] {hasMember} (e);
  \draw[oRelDir,emph] (i) to [bend left] node[oRlab] {hasMember} (d);

  \draw[oRelDir,emph,dashed] (e) to node[oRlab] {reference} (d);

\end{tikzpicture}


%%% Local Variables:
%%% mode: latex
%%% TeX-master: "../main"
%%% End:

  \caption{\label{uml-toplevel}
    \ow{InformationArtefact} and subclasses.}
\end{figure}


\begin{axioms}
  ~
  \begin{enumerate}[series=axioms,{label=(\arabic*)}]
  \item \ow{InformationArtefact} is a subclass of \ow{Entity}.
  \item \ow{Element}, \ow{Descriptor}, and \ow{InformationModel} are subclasses of \ow{InformationArtefact}.
  \item \ow{Element}, \ow{Descriptor}, and \ow{InformationModel} are pairwise disjoint.
  \item \ow{reference} is a relation to a \ow{Descriptor}.
  \item \ow{hasCharacteristic} is a subrelation of \ow{reference} that relates \ow{Descriptor}s.
  \end{enumerate}
\end{axioms}

\subsection{\ow{InformationModel}}

\ow{Element}s and \ow{Descriptor}s may be collected into and organized by \ow{InformationModel}s,
hence an \ow{InformationModel} represents a set of such \ow{InformationArtefact}s.
\ow{Element}s and \ow{Descriptor} are associcated with an \ow{InformationModel} using the \ow{memberOf}/\ow{hasMember} relations.



\subsection{Elements}
\label{sec:Elements}

\autoref{uml-elements} displays the class \ow{Element} and its subclasses.
\autoref{fig:graphics-node-shapes} shows the graphical language for describing \ow{Element}s.

\begin{figure}
  \centering
  \begin{tikzpicture}
    \tikzElementsAll
    \draw[oRelDir] (e) to [out=60 ,in=10 ,looseness=5] node[oRlab] {elementRelation}   (e);
  \end{tikzpicture}
  \caption{\label{uml-elements}
    \ow{Element} and its subclasses.}
\end{figure}

\begin{figure}
  \begin{center}
    \begin{tblr}{
        colspec = {Q[c]Q[c]Q[c]Q[c]Q[c]Q[c]},
        rowsep = 3pt
      }      
      \tikz\node[block]{}; &
      \tikz\node[connector]{}; &
      \tikz\node[terminal]{}; &
      \tikz\node[inputterminal]{}; &
      \tikz\node[outputterminal]{}; &
      \tikz\node[biterminal]{};
      \\
      \ow{Block} &
      \ow{Connector} &
      \ow{Terminal} &
      \ow{InputTerminal} &
      \ow{OutputTerminal} &
      \ow{BiTerminal}
      \\
    \end{tblr}
  \end{center}
\caption{\label{fig:graphics-node-shapes}All node shapes.}
\end{figure}


\subsubsection{\ow{Element}}

An \ow{Element} represents the specification of an asset. All assets
specifications are represented by an \ow{Element}.
%
An \ow{Element} is described through its relations (\autoref{sec:relations}) to other \ow{Element}s,
%
its \ow{Aspect} (\autoref{sec:aspect}),
%
%% its optional \ow{Context} (\autoref{sec:contexts}),
%
and it may be assigned \ow{Attribute}s (\autoref{sec:attributes}).
%

An \ow{Element} is either a \ow{Block}, a \ow{Connector} or a \ow{Terminal}.
Individuals rarely instantiate the class \ow{Element} directly, rather
they are stated to be an instance of a subclass of
\ow{Element}. (\ow{Element} can be considered an ``abstract class''.)



The permissive relationships for an \ow{Element} are controlled by its
type of subclass, and its \ow{Aspect}.
% ,and \ow{Context}.

An \ow{elementRelation} is a generic relation that relates \ow{Element}s.

\begin{axioms}
  ~
  \begin{enumerate}[resume=axioms,{label=(\arabic*)}]
  \item An \ow{Element} is either a \ow{Block}, a \ow{Connector}, or a \ow{Terminal}. 
  \item \ow{Block}, \ow{Connector}, and \ow{Terminal} are pairwise disjoint.
  \item \ow{elementRelation} is a relation between \ow{Element}s.
  \end{enumerate}
\end{axioms}

\subsubsection{\ow{Block}}

A \ow{Block} represents an abstraction of any entity at any
abstraction level. A \ow{Block} also represents a boundary that
separates what is internal and external to the
\ow{Block}. \ow{Element}s that are part of the \ow{Block} are internal
to the \ow{Block} and are not directly accessible to \ow{Element}s
outside (the boundary of) the \ow{Block}.
When a \ow{Block} represents a system, it can be natural to
consider the \ow{Block} as a processing black box that transforms input
to output, where the transformation process is detailed by the internals
of the \ow{Block}.

% \begin{example}
%   @@TODO?

%   Block represents something of interest to the SME by setting the
%   boundaries of anything which is convenient to treat as an
%   entity. This could be a whole industry plant, a pump system, or a
%   small location of interest.
% \end{example}


\subsubsection{\ow{Connector}}

A \ow{Connector} represents a topological connection between two
\ow{Block}s, and it will specify conditions on this connection, such
as the medium that is transferred between the \ow{Block}s.

A \ow{Connector} can be considered as a \ow{Block} with infinitely small
boundary, i.e., a \ow{Block} where there is no transformation of
the input to the output at the current abstraction level of the
\ow{Connector}. A \ow{Connector} may be further described by its
parts.

\subsubsection{\ow{Terminal}}

A \ow{Terminal} represents a point on a \ow{Block}'s boundary where
medium may pass, and it will specify conditions on the medium that may
pass through it.
%
A \ow{Terminal} describes the permissible input and/or input of
a \ow{Block}.
%
Any interaction with a \ow{Block} goes through a \ow{Terminal} of the
\ow{Block}.

A \ow{Terminal} depends on a \ow{Block}, hence a \ow{Terminal} cannot
exist without a \ow{Block}.


\ow{Block}s may interact by transferring media between them (\autoref{sec:media-transfer}).
%
A \ow{Terminal} may refer to \ow{direction} that specifies if the
\ow{Terminal} is set up to send or receive media.
%
A \ow{Terminal} that is specified to only receive input is called an \ow{InputTerminal}.
A \ow{Terminal} that is specified to only send output is called an \ow{OutputTerminal}.
A \ow{Terminal} that may both receive input and give output is called a \ow{BiTerminal}.

\begin{axioms}~
  \begin{enumerate}[resume=axioms,{label=(\arabic*)}]
  \item \ow{direction} is a subrelation of \ow{reference} that is functional relation and associates a \ow{Terminal} to a
    transfer direction.
  \item An \ow{InputTerminal} is a \ow{Terminal} with \ow{direction} \ow{input}.
  \item An \ow{OutputTerminal} is a \ow{Terminal} with \ow{direction} \ow{output}.
  \item A \ow{BiTerminal} is a \ow{Terminal} with \ow{direction} \ow{bidirectional}.
  \end{enumerate}
\end{axioms}

\subsubsection{Classifiers}
\label{sec:classifiers}

\ow{classifier} is a \ow{reference} from an \ow{Element},
such that $\ow{classifier}(a,b)$ expresses that
the \ow{Element} $a$ is classified by the \ow{Descriptor} $b$.
%
The relation is typically used to categorize \ow{Element}s according to
identifiers from standard reference data libraries or fixed
enumerations for the purpose of connecting to and reusing well-known and shared
identifiers and definitions.

Subrelations of \ow{classifier}
define for more specific classification relations.
We distinguish between
\ow{primaryClassifier}s that assign \emph{essential} descriptions or features to \ow{Element}s, and
\ow{auxClassifier}s (auxiliary) that assign \emph{inessential} descriptions or features to \ow{Element}s.

An \ow{Element} should have at least one \ow{primaryClassifier} that
identifies the logically necessary types of an \ow{Element}.
%
Additional classifiers of \ow{Element}s may be assigned using
\ow{auxClassifier}.

\begin{axioms}~
  \begin{enumerate}[resume=axioms,{label=(\arabic*)}]
  \item \ow{classifier} is a subrelation of \ow{reference} that relates an \ow{Element} to a \ow{Descriptor}.
  \item \ow{primaryClassifier} and \ow{auxClassifier} are subrelations of
    \ow{classifier}.
  \end{enumerate}
\end{axioms}



\subsection{Aspects}
\label{sec:aspect}

\ow{Element}s can be described from a specific point of view or
aspect, in the IMF language this is represented by \ow{Aspect}s.
%
An \ow{Aspect} is a \ow{Descriptor} specified by
one \ow{Modality},
one \ow{InformationDomain},
and one optional \ow{Interest}---all of which are also \ow{Descriptor}s.



\begin{figure}
\centering
\begin{tikzpicture}[
  every label/.append style={font=\sffamily\scriptsize}
  ]

  \node[oClass]                               (a)  {Aspect};
  \node[oClass, above right  = 1 and 4 of a]  (m)  {Modality};
  \node[oClass, below = of m]                 (e)  {InformationDomain};
  \node[oClass, below = of e]                 (i)  {Interest};

  \draw[oRelDir,emph] (a) -- node[oRlab]{hasCharacteristic} (m);
  \draw[oRelDir,emph] (a) -- node[oRlab]{hasCharacteristic} (e);
  \draw[oRelDir,emph] (a) -- node[oRlab]{hasCharacteristic} (i);

\end{tikzpicture}

%%% Local Variables:
%%% mode: latex
%%% TeX-master: "../main"
%%% End:

\caption{\label{uml-aspect}
\ow{Aspect}.}
\end{figure}


\ow{Modality} specifies the mode of existence of what is described.
There are two predefined \ow{Modality}-s: \ow{intended} and \ow{actual}.

\ow{Interest} specifies the modelling interest or context from which the
\ow{Element} is described. There are two predefined \ow{Interest}s:
\ow{projectLifecycle} and \ow{productLifecycle}

\ow{Aspect}s are associated with an \ow{InformationDomain} that
specifies what type of entities are in described by the \ow{Aspect}.
There are three predefined \ow{InformationDomain}s: \ow{activity}, \ow{implementation} and \ow{space}.

The various combinations of \ow{Modality}, \ow{InformationDomain}
and \ow{Interest} give rise to different \ow{Aspect}s;
these are listed in \autoref{fig:aspects}.

\begin{figure}
  \centering
  \begin{tabular}{llll}
    \ow{Modality} & \ow{Interest} & \ow{EntityType}      & \ow{Aspect}  \\ \hline
    \ow{intended} &               & \ow{activity}        & \ow{intendedActivity}       (\ow{functionAspect}) \\
    \ow{intended} &               & \ow{space}           & \ow{intendedSpace}          (\ow{locationAspect}) \\
    \ow{intended} &               & \ow{implementation}  & \ow{intendedImplementation} (\ow{productAspect}) \\
    \ow{actual}   &               & \ow{implementation}  & \ow{actualImplementation}   (\ow{installedAspect}) \\
    
    \ow{intended} & \ow{projectLifecycle} & \ow{activity}        & \ow{intendedProjectActivity} \\
    \ow{intended} & \ow{projectLifecycle} & \ow{space}           & \ow{intendedProjectSpace}    \\
    \ow{intended} & \ow{projectLifecycle} & \ow{implementation}  & \ow{intendedProjectImplementation} \\

    \ow{intended} & \ow{productLifecycle} & \ow{activity}        & \ow{intendedProductActivity} \\
    \ow{intended} & \ow{productLifecycle} & \ow{space}           & \ow{intendedProductSpace}    \\
    \ow{intended} & \ow{productLifecycle} & \ow{implementation}  & \ow{intendedProductImplementation}
  \end{tabular}
  \caption{Aspects}
  \label{fig:aspects}
\end{figure}


\begin{figure}
\centering
\begin{tabular}{lll}
  \ow{Aspect}/\ow{Entity type} filter & Colour name & Colour Code\\
  \hline
  No \ow{Aspect} & White & \myboxt{ffffff}\\
  Unspecified \ow{Aspect} & Grey & \myboxt{cccccc}\\
  %
  \ow{Activity} & Yellow & \myboxt{ffff00}\\
  \ow{Space} & Magenta & \myboxt{ff00ff}\\
  \ow{Implementation} & Cyan & \myboxt{00ffff}\\
\end{tabular}
\caption{\label{tbl:graphics-all-colours} Predefined colours for \ow{Entity type} filters and \ow{Aspect}s.}
\end{figure}


\begin{axioms}~
  \begin{enumerate}[resume=axioms,{label=(\arabic*)}]

  \item \ow{Aspect}, \ow{Modality}, \ow{Interest} and \ow{InformationDomain} are subclasses of \ow{Descriptor}.
    
  \item \ow{intended} and \ow{actual} are \ow{Modality}-s.
  \item \ow{projectLifecycle} and \ow{productLifecycle} are \ow{Interest}s.
  \item \ow{activity}, \ow{space} and \ow{implementation} are \ow{InformationDomain}s.

  \item An \ow{Aspect} is associated by one \ow{Modality}, optionally one \ow{Interest} and one \ow{InformationDomain} via the \ow{hasCharacteristic} relation.
  \item
    \ow{intendedActivity},
    \ow{intendedSpace},
    \ow{intendedImplementation},
    \ow{intendedProjectActivity},    
    \ow{intendedProjectSpace},   
    \ow{intendedProjectImplementation},
    \ow{intendedProductActivity},
    \ow{intendedProductSpace}
    and
    \ow{intendedProductImplementation}
    are \ow{Aspect}s and are specified as in \autoref{fig:aspects}.
  \end{enumerate}
\end{axioms}




\subsection{Aspect Elements}
\label{sec:aspect-elements}

An \ow{Element} may have at most one \ow{Aspect}.
%
An \ow{Element} that has an \ow{Aspect} is called an
\ow{AspectElement}.

It is convenient to be able to identify \ow{Element}s that carry
descriptions that are not of a single \ow{Aspect} and where the
\ow{Element}'s description is not analysed into separate
\ow{Aspect}s. Such \ow{Element}s are called \ow{NoAspectElement}s and
have explicitly no \ow{Aspect}.

%Technically this may also be stated by
%assigning the ``no value'' \ow{none} as its \ow{Aspect}.

Note that an \ow{Element} where its \ow{Aspect} is not known is not a
\ow{NoAspectElement}. Such an \ow{Element} is neither an
\ow{AspectElement} nor a \ow{NoAspectElement}, simply an \ow{Element}.

\ow{Aspect}s are used to control the permissive relationships an
\ow{Element} may have, in part by way of the  relations
\ow{intraAspectRelation} and \ow{interAspectRelation}.
%
An \ow{intraAspectRelation} relates \ow{Element}s that have the same
value assigned as its \ow{Aspect}, while an \ow{interAspectRelation}
relates \ow{Element}s that have different values.

%In both cases the
%value \ow{none} is included as a possible value for \ow{Aspect}.

\begin{figure}
  \begin{center}
    \begin{tblr}{
        colspec = {Q[c]Q[c]Q[c]Q[c]Q[c]Q[c]},
        rowsep = 3pt
      }
      \tikz\node[block,nonaspect]{}; &
      \tikz\node[block,unknown]{}; &
      \tikz\node[block,activity]{}; &
      \tikz\node[block,space]{}; &
      \tikz\node[block,implementation]{}; 
      %\\
      %\ow{ActivityBlock} &
      %\ow{ImplementationBlock} &
      %\ow{SpaceBlock} &
      %\ow{NonPerspectiveBlock} &
      %\ow{Block}
    \end{tblr}
  \end{center}
\caption{\label{fig:graphics-ex-coloured-blocks} Predefined colours demonstrated on \ow{Block}s.}
\end{figure}



\begin{figure}
\centering
\begin{tikzpicture}[
  every label/.append style={font=\sffamily\scriptsize}
  ]

  \node[oClass]                                (e)  {Element};
  \node[oClass, below left  = 1 and -.5 of e]  (ae) {AspectElement};
  \node[oClass, left = 2 of ae]                (a)  {Aspect};

  \node[oClass, below right = 1 and -.5 of e]   (ne) {NoAspectElement};
  %\node[oInd, right = 2 of ne,label=below:none] (none) {};
 
  \draw[oSpec] (ae) -- (e);
  \draw[oSpec] (ne) -- (e);

  \draw[oRelDir,emph] (ae) -- node[oRlab]{hasAspect} (a);
  %\draw[oRelDir,emph] (ne) -- node[oRlab]{hasAspect} (none);

  \draw[oRelDir,emph] (e) to [out=60 ,in=10 ,looseness=5] node[oRlab] {intraAspectRelation}   (e);
  \draw[oRelDir,emph] (e) to [out=120 ,in=170 ,looseness=5] node[oRlab] {interAspectRelation}   (e);


  % % aspect individuals: function, ...
  % \node[oInd, below right = .5  and -3 of a,label=below:function]  (a1) {};
  % \node[oInd, below right = 1   and -2 of a,label=below:location]  (a2) {};
  % \node[oInd, below right = 1.5 and -1 of a,label=below:product]   (a3) {};
  % \node[oInd, below right = 1   and  0 of a,label=below:installed] (a4) {};
  
  % \draw[oRel] (a) -- (a1);
  % \draw[oRel] (a) -- (a2);
  % \draw[oRel] (a) -- (a3);
  % \draw[oRel] (a) -- (a4);

\end{tikzpicture}

%%% Local Variables:
%%% mode: latex
%%% TeX-master: "../main"
%%% End:

\caption{\label{uml-aspectelement}
\ow{Aspect}s and \ow{Element}s.}
\end{figure}

\begin{axioms}~
  \begin{enumerate}[resume=axioms,{label=(\arabic*)}]
  \item \ow{hasAspect} is a subrelation of \ow{reference} that is a functional relation from \ow{Element} to
    \ow{Aspect}.
    %plus the value \ow{none}.
  \item An \ow{AspectElement} is an \ow{Element} with an \ow{Aspect}
    %(and not \ow{none})
    assigned to it by the relation \ow{hasAspect}.
  \item A \ow{NoAspectElement} is an \ow{Element} with
    % \ow{none} assigned to it by the relation \ow{hasAspect}
    no \ow{Aspect}.

  \item \ow{intraAspectRelation} and \ow{interAspectRelation} are subrelations of \ow{elementRelation}.
    
  \item If two \ow{Element}s are related by an
    \ow{intraAspectRelation} they must either share the same \ow{Aspect} or have no aspect.

  \item If two \ow{Element}s are related by an \ow{interAspectRelation} they cannot share the same \ow{Aspect}.

  \end{enumerate}
\end{axioms}


\subsection{Relations}
\label{sec:relations}

The IMF ontology contains a spesific set of relations for describing \ow{Element}s. The
main relational structures are specialization, topology, partonomy,
requirement--solution relationship, and relations between
\ow{AspectElement} that describe different aspects of the same entity.
All these relations are subrelations of \ow{elementRelation}, either directly or indirectly.

\autoref{fig:graphics-all-edges} shows the edges available for visualizing
relationships.

\begin{figure}
  \centering
\begin{tblr}{
  colspec = {QQ[c]QQ[c]},
  stretch = 0,
  rowsep = 10pt,
  column{2} = {leftsep=1pt,rightsep=2cm},
  %hlines = {red5, 1pt},
  %vlines = {red5, 1pt},
}
$\ow{connectedTo}(a,b)$  &
\begin{tikzpicture}[baseline=(current bounding box.center)]
  \node (a) at (0,0)  {$a$};
  \node (b) at (2,0)  {$b$};
  \draw (a) edge[connectedTo] (b);
\end{tikzpicture}
&
$\ow{transfersTo}(a,b)$  &
\begin{tikzpicture}[baseline=(current bounding box.center)]
  \node (a) at (0,0)  {$a$};
  \node (b) at (2,0)  {$b$};
  \draw (a) edge[transfersTo] (b);
\end{tikzpicture}
\\
$\ow{containedIn}(a,b)$  &
\begin{tikzpicture}[baseline=(current bounding box.center)]
  \node (a) at (0,0)  {$a$};
  \node (b) at (0,2)  {$b$};
  \draw (a) edge[containedIn] (b);
\end{tikzpicture}
&&
\\
$\ow{specializationOf}(a,b)$  &
\begin{tikzpicture}[baseline=(current bounding box.center)]
  \node (a) at (0,0)  {$a$};
  \node (b) at (0,2)  {$b$};
  \draw (a) edge[specializationOf] (b);
\end{tikzpicture}
&
$\ow{fulfills}(a,b)$  &
\begin{tikzpicture}[baseline=(current bounding box.center)]
  \node (a) at (0,0)  {$a$};
  \node (b) at (0,2)  {$b$};
  \draw (a) edge[fulfills] (b);
\end{tikzpicture}\\
$\ow{proxy}(a,b)$  &
\begin{tikzpicture}[baseline=(current bounding box.center)]
  \node (a) at (0,0)  {$a$};
  \node (b) at (1.5,1.5)  {$b$};
  \draw (a) edge[proxy] (b);
\end{tikzpicture}
&
$\ow{projection}(a,b)$  &
\begin{tikzpicture}[baseline=(current bounding box.center)]
  \node (a) at (0,0)  {$a$};
  \node (b) at (1.5,-1.5)  {$b$};
  \draw (a) edge[projection] (b);
\end{tikzpicture}\\
$\ow{equals}(a,b)$  &
\begin{tikzpicture}[baseline=(current bounding box.center)]
  \node (a) at (0,0)  {$a$};
  \node (b) at (1.5,1.5)  {$b$};
  \draw (a) edge[equals] (b);
\end{tikzpicture}
&&
\end{tblr}
  \caption{All edge types, with their preferred orientation indicated.}
  \label{fig:graphics-all-edges}
\end{figure}


\subsubsection{Specialization}

\ow{specializationOf} is a hierarchical relation and a subrelation of \ow{intraAspectRelation},
such that $\ow{specializationOf}(a,b)$ expresses that
the \ow{Element} $a$ is a specialization of the \ow{Element} $b$.  If
$a$ is a \ow{specializationOf} $b$, then $a$ inherits \emph{all} the
facts expressed for $b$. 

\ow{specializationOf} is used when specialized \ow{Element}s are
created from one or more generic \ow{Element}s,
%
or if the shared facts for multiple \ow{Element}s are to be refactored
into new generalized \ow{Element}(s). Using \ow{specializationOf} avoids
repetition of facts and makes models more compact, possibly at the
expense of added cognitive complexity to models.

\begin{axioms}~
  \begin{enumerate}[resume=axioms,{label=(\arabic*)}]
  \item \ow{specializationOf} is subrelation of \ow{intraAspectRelation}.
  \item \ow{specializationOf} is reflexive, anti-symmetric, and transitive.
  \item If $a$ is a \ow{specializationOf} $b$, then $a$ inherits all
    facts expressed for $b$.
  \item \ow{specializes} is the inverse relation of \ow{specializationOf}.
  \end{enumerate}
\end{axioms}




\subsubsection{Topology}
\label{sec:topology}

Topological relations describe how \ow{Element}s are
connected and interact.
Connected \ow{Element}s are described on a similar
level of abstraction.

The defining topology relation is \ow{connectedTo}.
%
\ow{connectedTo} is an associative relation between \ow{Element}s,
such that $\ow{connectedTo}(a,b)$ expresses that $a$ is topologically
connected to $b$.
%
\ow{adjacentTo} is a subrelation of \ow{connectedTo}, such
that $\ow{adjacentTo}(a,b)$ expresses that $a$ is directly
topologically connected to $b$. That means that there is no element
topologically ``in between'' $a$ and $b$.

Two ``shorthand'' relations are also defined:
%
\ow{hasTerminal} is a subrelation of \ow{adjacentTo} between \ow{Block}s and \ow{Terminal}s, while
%
\ow{hasConnector} is a subrelation of \ow{adjacentTo} between \ow{Terminal}s and \ow{Connector}s.

\begin{figure}
  \centering
  \begin{tikzpicture}[baseline=1cm, every node/.style={unknown}]
    \node[block]       (b1)   at (0,0)  {$a$};
    \draw[connectedTo] (b1) -- ++(-2,-.7) node[terminal] (b1it1) {};
    \draw[connectedTo] (b1) -- ++(-2, .7) node[terminal] (b1it2) {};
    \draw[dotted] (b1it1) -- (b1it2);
    \draw[connectedTo] (b1) -- ++( 2,-.7) node[terminal] (b1ot1) {};
    \draw[connectedTo] (b1) -- ++( 2, .7) node[terminal] (b1ot2) {};
    \draw[dotted] (b1ot1) -- (b1ot2);
    
    \node[block]       (b2)   at (8,2)  {};
    \draw[connectedTo] (b2) -- ++(-2,-.7) node[terminal] (b2it1) {};
    \draw[connectedTo] (b2) -- ++(-2, .7) node[terminal] (b2it2) {};
    \draw[dotted] (b2it1) -- (b2it2);
    \draw[connectedTo] (b2) -- ++( 2,-.7) node[terminal] (b2ot1) {};
    \draw[connectedTo] (b2) -- ++( 2, .7) node[terminal] (b2ot2) {};
    \draw[dotted] (b2ot1) -- (b2ot2);

    \node[block]       (b3)   at (8,-2)  {$b$};
    \draw[connectedTo] (b3) -- ++(-2,-.7) node[terminal] (b3it1) {};
    \draw[connectedTo] (b3) -- ++(-2, .7) node[terminal] (b3it2) {};
    \draw[dotted] (b3it1) -- (b3it2);
    \draw[connectedTo] (b3) -- ++( 2,-.7) node[terminal] (b3ot1) {};
    \draw[connectedTo] (b3) -- ++( 2, .7) node[terminal] (b3ot2) {};
    \draw[dotted] (b3ot1) -- (b3ot2);
    
    \node[connector]    (c1) at (4,-.9)  {};
    \node[connector]    (c2) at (4, .9)  {};

    \draw[connectedTo] (b1ot1) -- (c1) -- (b3it2);
    \draw[connectedTo] (b1ot2) -- (c2) -- (b2it1);

    \node[fit=(b1it1)(b1ot2),draw, inner sep=10pt,blue!30,fill=none] {};
    %\node[fit=(b2it1)(b2ot2),draw, inner sep=10pt,blue!30] {};
    \node[fit=(b3it1)(b3ot2),draw, inner sep=10pt,blue!30,fill=none] {};

    \node[fit=(b1ot1)(c1)(b3it2),draw, inner sep=5pt,green!40,fill=none] {};
    %\node[fit=(b1ot2)(c2)(b2it1),draw, inner sep=5pt,green!40] {};

    \draw[connectedTo,pink,very thick] (b1) |- node[fill=none,above,color=black]{$p$} (c2);
    \draw[connectedTo,pink,very thick] (b1) |- node[fill=none,below,color=black]{$q$} (b3);
  \end{tikzpicture}

  \caption{Pattern for topologically connected \ow{Element}s:
    % 
    \ow{Block}s are through their \ow{Terminals} (boxed in blue)
    connected via \ow{Connector}s (boxed in green).
    %
    A \ow{connectedTo} relationship between a \ow{Block} and a \ow{Connector} (the edge labelled $p$) entails the existence of a \ow{Terminal} between them.
    %
    A \ow{connectedTo} relationship between two \ow{Block}s (the edge
    labelled $p$) entails the existence of a pattern of the form of
    the green boxed pattern between them.}
  \label{fig:topology-pattern}
\end{figure}

The difference between these topological relations is that
\ow{connectedTo} allows for expressing the topology of \ow{Element}s
between \emph{any} kind of \ow{Element}s, e.g., but not limited to,
two \ow{Block}s, or a \ow{Block} and a \ow{Connector}, while the use
of \ow{adjacentTo} and its subrelations is more restricted.
However, a \ow{connectedTo} relationship
is required to be 
realizable by a
chain of \ow{adjacentTo} relationships.
%
This is to force the following reoccurring pattern in all topological chains of
\ow{Element}s:
$$\ldots {}
\rightarrow \ow{Block}
\rightarrow \ow{Terminal}
\rightarrow \ow{Connector}
\rightarrow \ow{Terminal}
\rightarrow \ow{Block}
\rightarrow {} \ldots$$
%
Note however that each \ow{Element} in the chain is not necessarily explicitly
expressed, but may be left implicit
%, but all \ow{Element}s do need to exist 
%
(\autoref{fig:topology-pattern}).
%
This can be useful in the case of representing incomplete information
which may occur early
in the modelling process or when
the information to be 
modelled is at a very high
abstraction level where, e.g., the conditions for
\ow{Terminal}s and \ow{Connector}s are not known, but it is clear
which \ow{Block}s are connected.


\begin{figure}
\centering
\begin{tikzpicture}

  \tikzElementsBasic

  \draw[oRelDir,emph] (b) -- node[oRlab] {hasTerminal} (t);
  \draw[oRelDir,emph] (t) -- node[oRlab] {hasConnector} (c);

  \draw[oRelDir,emph] (e) to [out=60 ,in=10 ,looseness=4] node[oRlab] {adjacentTo}   (e);
  \draw[oRelDir,emph] (e) to [out=170,in=120,looseness=4] node[oRlab] {connectedTo}  (e);

  \draw[oRelDir,emph] (b) to [bend right] node[oRlab,below] {connectedThrough}  (c);

\end{tikzpicture}


\caption{\label{uml-topology} Topology relations.}
\end{figure}



The following axioms are defined for topological relations (\autoref{uml-topology}):
\begin{axioms}~
  \begin{enumerate}[resume=axioms,{label=(\arabic*)}]
  \item \ow{connectedTo} is an \ow{intraAspectRelation}.
  \item \ow{adjacentTo} is a subrelation of \ow{connectedTo}. 
  \item \ow{connectedTo} and \ow{adjacentTo} are symmetric.
  \item \ow{hasTerminal} is a subrelation of
    \ow{adjacentTo} that associates a \ow{Block} with its
    \ow{Terminal}s.
  \item \ow{connectedThrough} is a relation from a \ow{Block} to a
    \ow{Connector} such that the relation is equal to the composition
    of \ow{hasTerminal} and \ow{hasConnector}.
  \item \ow{hasConnector} is a subrelation of \ow{adjacentTo}
    that associates a \ow{Terminal} to its \ow{Connector}s.
  \item \ow{Block}s and \ow{Connector}s may be \ow{adjacentTo} to only \ow{Terminal}s.
  \item \ow{Element}s may be \ow{connectedTo} any number of \ow{Element}s.
  \item A \ow{Block} may be \ow{adjacentTo} any number of \ow{Terminal}s.
  \item A \ow{Terminal} must be \ow{adjacentTo}  one \ow{Block}.
    % \item A \ow{Terminal} may be \ow{adjacentTo} not more than one \ow{Connector}.
  \item A \ow{Connector} must be \ow{adjacentTo} exactly two \ow{Terminal}s. 
  \item \label{ax:topology-realisable}A \ow{connectedTo} relationship must be realizable by a sequence
      of \ow{adjacentTo} relationships.
  \end{enumerate}
\end{axioms}



\subsubsection{Partonomy}

Partonomy relations relate \ow{Element}s at different levels of
abstraction.
%
An \ow{Element} can be described by relating it to its \emph{parts},
which are \ow{Elements} described at a lower level of abstraction, and
by it relating to the \emph{whole} it is part of, which is an
\ow{Element} at a higher level of abstraction.
%
This recursive breakdown structure allows \ow{Element}s to be
described at incremental finer---or coarser, level of detail.

The defining partonomy relation is \ow{partOf}.
%
\ow{partOf} is a hierarchical relation between \ow{Element}s, such
that $\ow{partOf}(a,b)$ expresses that $a$ is part of $b$.  In such a
relationship, we call $a$ ``the part'' or ``child'', while $b$ is
called ``the whole'' or ``parent''.
%
Any \ow{Element} is a \ow{partOf} at most one other \ow{Element};
the \ow{partOf} hierarchy forms a tree-structured breakdown.
%
\ow{hasPart} is the inverse relation of \ow{partOf}.

\ow{containedIn} is the transitive closure of \ow{partOf}, and
\ow{contains} is the transitive closure of \ow{containedIn}.


\begin{figure}
  \centering
  \begin{tikzpicture}[baseline=1cm, every node/.style={unknown}]

    \node[block]       (b1)   at (0,0)  {$b$};
    \draw[connectedTo] (b1) -- ++(-2,-.7) node[terminal] (b1it1) {};
    \draw[connectedTo] (b1) -- ++(-2, .7) node[terminal] (b1it2) {};
    \draw[dotted] (b1it1) -- (b1it2);
    \draw[connectedTo] (b1) -- ++( 2,-.7) node[terminal] (b1ot1) {};
    \draw[connectedTo] (b1) -- ++( 2, .7) node[terminal] (b1ot2) {};
    \draw[dotted] (b1ot1) -- (b1ot2);

    %% parts step 1
    
    \node[block]       (b2)   at (-5,-3)  {};
    \draw[connectedTo] (b2) -- ++(-2,-.7) node[terminal] (b2it1) {};
    \draw[connectedTo] (b2) -- ++(-2, .7) node[terminal] (b2it2) {};
    \draw[dotted] (b2it1) -- (b2it2);
    \draw[connectedTo] (b2) -- ++( 2,-.7) node[terminal] (b2ot1) {};
    \draw[connectedTo] (b2) -- ++( 2, .7) node[terminal] (b2ot2) {};
    \draw[dotted] (b2ot1) -- (b2ot2);

    \draw[connectedTo] (b2ot1) -- ++(2, 0) node[connector] (c1)  {};
    \draw[connectedTo] (b2ot2) -- ++(2, 0) node[connector] (c2)  {};

    \node[block]       (b3)   at (5,-3)  {};
    \draw[connectedTo] (b3) -- ++(-2,-.7) node[terminal] (b3it1) {};
    \draw[connectedTo] (b3) -- ++(-2, .7) node[terminal] (b3it2) {};
    \draw[dotted] (b3it1) -- (b3it2);
    \draw[connectedTo] (b3) -- ++( 2,-.7) node[terminal] (b3ot1) {};
    \draw[connectedTo] (b3) -- ++( 2, .7) node[terminal] (b3ot2) {};
    \draw[dotted] (b3ot1) -- (b3ot2);

    \draw[connectedTo] (b3it1) -- ++(-2, 0) node[connector] (c3)  {};
    \draw[connectedTo] (b3it2) -- ++(-2, 0) node[connector] (c4)  {};

    \draw[dashed] (c1) -- (c3);
    \draw[dashed] (c2) -- (c4);

    %% parts step 2

    \node[block]       (b4)   at (-5,-5.5)  {};
    \draw[connectedTo] (b4) -- ++(-2,-.7) node[terminal] (b4it1) {};
    \draw[connectedTo] (b4) -- ++(-2, .7) node[terminal] (b4it2) {};
    \draw[dotted] (b4it1) -- (b4it2);
    \draw[connectedTo] (b4) -- ++( 2,-.7) node[terminal] (b4ot1) {};
    \draw[connectedTo] (b4) -- ++( 2, .7) node[terminal] (b4ot2) {};
    \draw[dotted] (b4ot1) -- (b4ot2);

    \draw[connectedTo] (b4ot1) -- ++(2, 0) node[connector] (c1)  {};
    \draw[connectedTo] (b4ot2) -- ++(2, 0) node[connector] (c2)  {};

    \node[block]       (b5)   at (5,-5.5)  {};
    \draw[connectedTo] (b5) -- ++(-2,-.7) node[terminal] (b5it1) {};
    \draw[connectedTo] (b5) -- ++(-2, .7) node[terminal] (b5it2) {};
    \draw[dotted] (b5it1) -- (b5it2);
    \draw[connectedTo] (b5) -- ++( 2,-.7) node[terminal] (b5ot1) {};
    \draw[connectedTo] (b5) -- ++( 2, .7) node[terminal] (b5ot2) {};
    \draw[dotted] (b5ot1) -- (b5ot2);

    \draw[connectedTo] (b5it1) -- ++(-2, 0) node[connector] (c3)  {};
    \draw[connectedTo] (b5it2) -- ++(-2, 0) node[connector] (c4)  {};
    
    \draw[dashed] (c1) -- (c3);
    \draw[dashed] (c2) -- (c4);

    %% surrounding boxes

    \node[fit=(b1it1)(b1it2),draw, inner sep=5pt,gray!30,fill=none] (b1sit) {};
    \node[fit=(b1ot1)(b1ot2),draw, inner sep=5pt,gray!30,fill=none] (b1sot) {};

    \node[fit=(b2)(b3)(b4)(b5)(b2ot2)(b5it1),draw, inner sep=10pt,gray!30,fill opacity=0.8, fill=white] (b1parts) {};

    \node[fill=none] at (b1parts) {$C$};
    
    \node[fit=(b2it2)(b4it1),draw, inner sep=5pt,gray!30,fill=none] (b1partsit) {};
    \node[fit=(b3ot2)(b5ot1),draw, inner sep=5pt,gray!30,fill=none] (b1partsot) {};
    
    \draw[containedIn] (b1parts)   -- (b1);
    \draw[containedIn] (b1partsit.north) -- (b1sit.west);
    \draw[containedIn] (b1partsot.north) -- (b1sot.east);
    
  \end{tikzpicture}

  \caption{Pattern for breakdown hierarchies. The \ow{Block} $b$ is
    broken down to a set of \ow{Element}s (boxed by $C$) that are all
    \ow{partOf} $b$ and form a legal topological structure, and is
    such that \ow{adjacentTo} relationships from a part of
    $b$ can only go to other parts of $b$ or to the parts of $b$'s
    \ow{Terminal}s. All the \ow{Terminal}s that connect into an
    \ow{Element} of $C$ (and not between \ow{Element}s of $C$) must be
    \ow{partOf} one of $b$'s \ow{Terminal}s.
    %
    Note that the surrounding boxes used in the diagram are
    simplifications; relations must go between \ow{Element}s and not
    sets of \ow{Element}s. }
  \label{fig:partonomy-pattern}
\end{figure}


The parts of an \ow{Element} must form a legal topological structure
as described in \autoref{sec:topology}. \ow{Element}s can have any
kind of \ow{Element}s as parts, except \ow{Terminal} whose parts must
be \ow{Terminal}s. The parts of an \ow{Element} are internal to the
\ow{Element} and may not directly connect to external \ow{Element}s,
except to \ow{Element}s that are \ow{partOf} the \ow{Terminal} of the
parent \ow{Element} (\autoref{fig:partonomy-pattern}).


\begin{figure}
\centering
\begin{tikzpicture}

  \tikzElementsBasic

  \draw[oRelDir,emph] (b) to[bend left]  node[oRlab] {contains} (e.west);
  \draw[oRelDir,emph] (c) to[bend right] node[oRlab] {contains} (e.east);
  \draw[oRelDir,emph] (t) to[loop,out=220, in=320,looseness=5] node[oRlab,below] {contains} (t);

  \draw[oRelDir,emph] (e) to [out=60 ,in=10 ,looseness=4] node[oRlab] {partOf}   (e);
  \draw[oRelDir,emph] (e) to [out=170,in=120,looseness=4] node[oRlab] {containedIn}  (e);

\end{tikzpicture}

\caption{\label{uml-partonomy}
Partonomy relations.}
\end{figure}

The following axioms are defined for partonomy relations (\autoref{uml-partonomy}):
\begin{axioms}~
  \begin{enumerate}[resume=axioms,{label=(\arabic*)}]
  \item \ow{containedIn} is an \ow{intraAspectRelation}.
  \item \ow{containedIn} is transitive.
  \item The inverse relation of \ow{containedIn} is \ow{contains}. 
  \item \ow{partOf} is a (non-transitive) subrelation of \ow{containedIn}. 
  \item \ow{partOf} is functional.
  \item \ow{partOf} is irreflexive.
  \item \ow{hasPart} is the inverse relation of \ow{partOf}.
  \item A \ow{containedIn} relationship must be realized by a chain of \ow{partOf} relationships.
  \item A \ow{Terminal} can only have \ow{Terminal}s as parts. 
  \item If two \ow{Element}s are \ow{adjacentTo} each other either,
    then they are each \ow{partOf} two \ow{Element}s that are
    either equal or they are also \ow{adjacentTo} each other.
  \end{enumerate}
\end{axioms}


\subsubsection{Media transfer}
\label{sec:media-transfer}

Topologically connected \ow{Block}s may interact by transferring media
between them.  The media transfer must respect the \ow{direction} and
media specified for the \ow{Terminal}s involved in the transfer.

The defining relation for media transfer is \ow{transfersTo}.
%
\ow{transfersTo} is a relation between two \ow{Terminal}s or between
two \ow{Block}s such that $\ow{transfersTo}(a,b)$ expresses that (the
source \ow{Element}) $a$ transfers to (the target \ow{Element}) $b$.
%
A \ow{transfersTo} relation between two \ow{Block}s must be
realized by a \ow{transfersTo} relationship between \ow{Terminal}s
that belong to the \ow{Block}s.
%
\ow{transfersTo} is a non-symmetric subrelation of \ow{connectedTo}:
only \ow{Element}s that are \ow{connectedTo} each other can
\ow{transfersTo} each other, with the following restrictions:
%
an \ow{OutputTerminal} may only \ow{transfersTo} an
\ow{InputTerminal}, and a \ow{BiTerminal} may only \ow{transfersTo} a
\ow{BiTerminal}. An \ow{InputTerminal} cannot be the source
\ow{Element} in a \ow{transfersTo} relationship, it may only be the target.

\ow{medium} is a subrelation of \ow{classifier} that indicates the
medium that a \ow{Terminal} or \ow{Connector} handles.

\begin{figure}
\centering
\begin{tikzpicture}

  \tikzElementsAll

  \draw[oRelDir,emph] (b) to [out=180,in=120,looseness=4] node[oRlab,left] {transfersTo}  (b);
  \draw[oRelDir,emph] (t) to [out=180,in=120,looseness=4] node[oRlab,left] {transfersTo}  (t);
  \draw[oRelDir,emph] (ot) to [bend right] node[oRlab,below] {transfersTo}  (it);
  

\end{tikzpicture}

\caption{\label{uml-transfers}
  Media transfer relations.}
\end{figure}

The following axioms are defined for media transfer relations (\autoref{uml-transfers}):
\begin{axioms}~
  \begin{enumerate}[resume=axioms,{label=(\arabic*)}]
  \item \ow{medium} is a subrelation of \ow{classifier} that relates a
    \ow{Terminal} or a \ow{Connector} to an external medium classifier.
  \item \ow{transfersTo} is a relation between \ow{Terminal}s or between \ow{Block}s.
  \item \ow{transfersTo} is a subrelation of \ow{connectedTo}.
  \item \ow{transfersTo} is irreflexive.
  \item \ow{OutputTerminal}s may only \ow{transfersTo} \ow{InputTerminal}s.
  \item \ow{BiTerminal}s may only \ow{transfersTo} \ow{BiTerminal}s.
  \item \ow{InputTerminal}s may not \ow{transfersTo}.
  \item A \ow{transfersTo} relationship between \ow{Block}s must be
    realized by a \ow{transfersTo} relationships between
    \ow{Terminal}s.
  \end{enumerate}
\end{axioms}

\subsection{\ow{AspectElement} relations}
\label{sec:owasp-relat}


Descriptions of different aspects of entities are represented by
different \ow{AspectElement}s.
%
\ow{AspectElement} relations are relations that relate
\ow{AspectElement} which represent different \ow{Aspect} of the same
entity.

An entity may only be represented by at most one \ow{AspectElement}
for each \ow{Aspect}, and one \ow{NoAspectElement}.

\begin{figure}
  \centering
  \begin{tikzpicture}

    \node[block,function]  (f) at (-2,  2)  {$f$};
    \node[block,location]  (l) at ( 2 , 2)  {$l$};
    \node[block,product]   (p) at (-2 ,-2)  {$p$};
    \node[block,installed] (i) at ( 2 ,-2)  {$i$};
    
    \draw[proxy] (f) -- (l) -- (i) -- (p) -- (f);

    \draw[proxy]  (f) to[bend left=40] (i);
    \draw[proxy]  (l) to[bend left=40] (p);

    \node[block]      (n) at (0,0)  {$n$};
    \draw[projection] (n) -- (f);
    \draw[projection] (n) -- (l);
    \draw[projection] (n) -- (p);
    \draw[projection] (n) -- (i);

  \end{tikzpicture}
  \caption{Example of cluster of \ow{proxy} \ow{Elements}. The
    \ow{AspectElement}s $f$, $l$, $p$ and $i$ are all \ow{proxy} related, and
    \ow{projection}s of the \ow{NoAspectElement} $n$.}
  \label{fig:proxy-elements}
\end{figure}

There are two \ow{AspectElement} relations.
%
\ow{projection} is the relation between a \ow{NoAspectElement} and an
\ow{AspectElement}, such that $\ow{projection}(a,b)$ expresses that
$a$ is an \ow{NoAspectElement} representation of an entity and $b$
(called the projection of $a$) is an \ow{AspectElement}
representations of the same entity.  \ow{proxy} is the relation
between two \ow{AspectElement}s, such that $\ow{proxy}(a,b)$ expresses
that $a$ and $b$ are two different \ow{AspectElement}s that are
\ow{projection}s of the same \ow{NoAspectElement} (\autoref{fig:proxy-elements}).


The following axioms are defined for \ow{AspectElement} relations:
\begin{axioms}~
  \begin{enumerate}[resume=axioms,{label=(\arabic*)}]
  \item \ow{projection} is an inverse functional
    \ow{interAspectRelation} relation from a \ow{NoAspectElement} to
    \ow{AspectElement}.
  \item \ow{proxy} is a \ow{interAspectRelation} between two different
    \ow{AspectElement}s that are \ow{projection}s of the same
    \ow{NoAspectElement}.
  \end{enumerate}
\end{axioms}

\subsubsection{Requirement--Solution relations}

Requirement--Solution relations are used to express that \ow{Element}s
may express requirements that other \ow{Element} may satisfy and
provide solution for.

\ow{fulfilledBy} is a hierarchical relation between \ow{Element}, such
that $\ow{fulfilledBy}(a,b)$ expresses that $b$ is intended as a
solution to the requirements posited by $a$.

The following axioms are defined for Requirement--Solution relations:
\begin{axioms}~
  \begin{enumerate}[resume=axioms,{label=(\arabic*)}]
  \item \ow{fulfilledBy} is a subrelation of \ow{elementRelation}.
  \item \ow{fulfills} is the inverse relation of \ow{fulfilledBy}.
  \end{enumerate}
\end{axioms}



\subsection{Attributes}
\label{sec:attributes}

\ow{Attribute}s and \ow{AttributeGroup}s are \ow{Descriptor}s for assigning data values to \ow{Element}s.
%
The vocabulary for expressing \ow{Attribute}s is shown in \autoref{uml-attribute}.
%
An \ow{Attribute} can be specified with a \ow{predicate}, \ow{uom}
(unit of measure) and a \ow{value}.

\ow{Attribute}s may be grouped into
\ow{AttributeGroup}s, using the relation \ow{memberOf} or its inverse relation \ow{hasMember}.
$\ow{memberOf}(a,b)$ expresses that $a$ is member of $b$.

Both \ow{Attribute}s and
\ow{AttributeGroup}s can be qualified by different \ow{classifier}s (see
\autoref{sec:classifiers}) as a means to describe them.

\begin{axioms}~
  \begin{enumerate}[resume=axioms,{label=(\arabic*)}]
  \item \ow{hasAttribute} is a subrelation of \ow{reference} between \ow{Element} and \ow{Attribute}.
  \item \ow{uom} is a subrelation of \ow{reference} from \ow{Attribute}.
  \item \ow{value} is a datatype relation from \ow{Attribute} to a value.
  \item \ow{predicate} is a subrelation of \ow{classifier} from an \ow{Attribute}.
  \item An \ow{Attribute} can only be member of an \ow{AttributeGroup}, and
    an \ow{AttributeGroup} can only have \ow{Attribute}s as members.
  \end{enumerate}
\end{axioms}



\begin{figure}
\centering
\begin{tikzpicture}

  \tikzElement

  \node[oClass, below right = 0 and 3 of e] (atr) {Attribute};
  \node[oClass, above = 1.5 of atr] (atrg) {AttributeGroup};

  \node[oExt, right = 1.5 of atrg] (qual) {};
  \node[oExt, below =     of qual] (pred) {};
  \node[oExt, below = 0.5 of pred] (uom) {};
  \node[oInd, below = 0.5 of uom] (val) {};

  \draw [oRelDir] (e) to[] node[oRlab] {hasAttribute} (atr);
  \draw [oRelDir] (e) to[] node[oRlab] {hasAttribute} (atrg);
  \draw [oRelDir] (atrg) to[] node[oRlab] {hasAttribute} (atr);

  \draw [oRelDir] (atrg) to[] node[oRlab] {qualifier} (qual);
  \draw [oRelDir] (atr)  to[] node[oRlab] {qualifier} (qual);
  
  \draw [oRelDir] (atr) to[] node[oRlab] {uom} (uom);
  \draw [oRelDir] (atr) to[] node[oRlab] {predicate} (pred);
  \draw [oRelDir] (atr) to[] node[oRlab] {value} (val);
  

\end{tikzpicture}

%%% Local Variables:
%%% mode: latex
%%% TeX-master: "../main"
%%% End:

\caption{\label{uml-attribute}
\ow{Attribute} and \ow{AttributeGroup}s.}
\end{figure}



\subsection{Other}

The ontology contains a few other relations to describe \ow{Entity}-s:

\begin{itemize}
\item \ow{symbol} is a relation to associate a graphical representation, e.g., a diagram icon, with an \ow{Entity}.

\item \ow{color} is an annotation relation that is used to associate a resource with a color for use in visualizations.

\item \ow{notation} is a relation that is used to associate an \ow{Element} with an reference designation system (RDS) code.

\item \ow{prefix} is an annotation relation that is used to associate an \ow{Aspect} with its RDS prefix.

\end{itemize}






%\subsection{Classifiers}
%\label{sec:classifiers}
% medium: list 4 highlevel medium



\end{document}

TODO:

- new section on classifiers, listing values:
  - media
  - attribute classifiers
  - examples of classifiers.
  
- fulfilledBy, partial

- search for @TODO
- move intro text to chap 3?
- include examples?
- what does modellers need to know
- what does rp need to know
- look to
  - fol,
  - visualization,
  - ontology,
  - old ch 4
  - do we need a node shape for element? dashed rectangle/ellipse?
  - do we need a node shape for external classifier? dashed rectangle/ellipse?
- media
- differences from previous version:
  - connection points are renamed to connectors
  - breakdown points are removed
  - aspects are renamed
  - visualization is changed
  - inter-aspect relations are simplified to proxy
  - an aspect for no aspect is introduced
- RDL
  - subsection somewhere in intro
    - media
- model
- identifiers
- templas
- sameas

%%% Local Variables:
%%% mode: latex
%%% TeX-master: "main"
%%% End:
